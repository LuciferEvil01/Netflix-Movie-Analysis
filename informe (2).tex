\documentclass{article}
\usepackage[utf8]{inputenc}
\usepackage{graphicx}
\usepackage{amsmath}
\usepackage{hyperref}
\usepackage{booktabs}
\usepackage{float}

\title{Análisis Estadístico de un Dataset}
\author{Tu Nombre}
\date{\today}

\begin{document}

\maketitle

\tableofcontents
\newpage

% --- Sección 1: Introducción ---
\section{Introducción}
\subsection{Contexto del Problema}
Breve descripción del contexto y los objetivos del análisis. ¿Por qué es importante este dataset? ¿Qué preguntas se buscan responder?

\subsection{Descripción del Dataset}
Información general sobre el dataset:
\begin{itemize}
    \item Fuente de los datos.
    \item Número de filas y columnas.
    \item Variables principales y su tipo (numérica, categórica, etc.).
\end{itemize}

% --- Sección 2: Análisis Exploratorio de Datos (EDA) ---
\section{Análisis Exploratorio de Datos (EDA)}
\subsection{Limpieza de Datos}
\begin{itemize}
    \item Manejo de valores faltantes.
    \item Tratamiento de outliers.
    \item Corrección de inconsistencias.
\end{itemize}

\subsection{Análisis Descriptivo}
    \subsubsection{Estadísticas descriptivas (media, mediana, desviación estándar, etc.).}


\begin{table}[H]
    \centering
    \begin{tabular}{|l|r|}
    \hline
    \textbf{Estadística} & \textbf{Valor} \\
    \hline
    Duración Media & 99.528 \\
    Duración Mediana & 98.000 \\
    Moda de la Duración & 90.000 \\
    Varianza (min$^2$) & 804.816 \\
    Desviación Estándar & 28.369 \\
    Percentil 25 & 87.000 \\
    Percentil 50 (mediana) & 98.000 \\
    Percentil 75 & 114.000 \\
    \hline
    \end{tabular}
    \caption{Duración de Películas}
    \label{tab:estadisticas_duracion_peliculas}
\end{table}

\begin{table}[H]
    \centering
    \begin{tabular}{|l|r|}
    \hline
    \textbf{Estadística} & \textbf{Valor} \\
    \hline
    Duración Media & 1.765 \\
    Duración Mediana & 1.000 \\
    Moda de la Duración & 1.000 \\
    Varianza (\#temporadas$^2$) & 2.505 \\
    Desviación Estándar & 1.583 \\
    Percentil 25 & 1.000 \\
    Percentil 50 (mediana) & 1.000 \\
    Percentil 75 & 2.000 \\
    \hline
    \end{tabular}
    \caption{Duración de Series de TV}
    \label{tab:estadisticas_duracion_series}
\end{table}

\begin{table}[H]
    \centering
    \begin{tabular}{|l|r|}
    \hline
    \textbf{Estadística} & \textbf{Valor} \\
    \hline
    Mediana & 2017 \\
    Moda & 2018 \\
    Varianza & 78 \\
    Desviación Estándar & 9 \\
    Percentil 25 & 2013 \\
    Percentil 50 (mediana) & 2017 \\
    Percentil 75 & 2019 \\
    \hline
    \end{tabular}
    \caption{Año de Estreno}
    \label{tab:estadisticas_estreno}
\end{table}

\begin{table}[H]
    \centering
    \begin{tabular}{|l|r|}
    \hline
    \textbf{Estadística} & \textbf{Valor} \\
    \hline
    Mediana & 2019 \\
    Moda & 2019 \\
    Varianza & 2 \\
    Desviación Estándar & 2 \\
    Percentil 25 & 2018 \\
    Percentil 50 (mediana) & 2019 \\
    Percentil 75 & 2020 \\
    \hline
    \end{tabular}
    \caption{Año de Adición a  Netflix}
    \label{tab:estadisticas_adicion}
\end{table}
    \subsubsection{Visualizaciones}

\begin{figure}[H]
    \centering
    \includegraphics[width=\textwidth]{Graphs/directores_peliculas.png}
    \caption{Directores de Películas Más Comunes}
    \label{fig:peliculas_duracion}
\end{figure}

\begin{figure}[H]
    \centering
    \includegraphics[width=\textwidth]{Graphs/directores_series.png}
    \caption{Directores de Series Más Comunes}
    \label{fig:peliculas_duracion}
\end{figure}

\begin{figure}[H]
    \centering
    \includegraphics[width=\textwidth]{Graphs/categorias_peliculas.png}
    \caption{Categorías Más Comunes en Películas}
    \label{fig:categorías_peliculas}
\end{figure}

\begin{figure}[H]
    \centering
    \includegraphics[width=\textwidth]{Graphs/categorias_series.png}
    \caption{Categorías Más Comunes en Series de TV}
    \label{fig:categorías_series}
\end{figure}

\begin{figure}[H]
    \centering
    \includegraphics[width=\textwidth]{Graphs/evolucion_contenido.png}
    \caption{Evolución del Contenido}
    \label{fig:evolucion_contenido}
\end{figure}

\begin{figure}[H]
    \centering
    \includegraphics[width=\textwidth]{Graphs/paises_peliculas.png}
    \caption{Países con más películas}
    \label{fig:peliculas_paises}
\end{figure}

\begin{figure}[H]
    \centering
    \includegraphics[width=\textwidth]{Graphs/paises_series.png}
    \caption{Países con más series}
    \label{fig:peliculas_paises}
\end{figure}

\begin{figure}[H]
    \centering
    \includegraphics[width=\textwidth]{Graphs/conteo_rating_peliculas.png}
    \caption{Conteo de Ratings para Películas}
    \label{fig:conteo_ratings_peliculas}
\end{figure}

\begin{figure}[H]
    \centering
    \includegraphics[width=\textwidth]{Graphs/conteo_rating_series.png}
    \caption{Conteo de Ratings para Series de TV}
    \label{fig:conteo_ratings_series}
\end{figure}

\begin{figure}[H]
    \centering
    \includegraphics[width=\textwidth]{Graphs/dist_rating_year_peliculas.png}
    \caption{Distribución de Ratings por Año de Adición en Películas}
    \label{fig:distribucion_ratings_peliculas}
\end{figure}

\begin{figure}[H]
    \centering
    \includegraphics[width=\textwidth]{Graphs/dist_rating_year_series.png}
    \caption{Distribución de Ratings por Año de Adición en Series de TV}
    \label{fig:distribucion_ratings_series}
\end{figure}

\begin{figure}[H]
    \centering
    \includegraphics[width=\textwidth]{Graphs/actores_comunes_peliculas.png}
    \caption{Actores Más Comunes en Películas}
    \label{fig:actores_peliculas}
\end{figure}

\begin{figure}[H]
    \centering
    \includegraphics[width=\textwidth]{Graphs/actores_comunes_series.png}
    \caption{Actores Más Comunes en Series de TV}
    \label{fig:actores_series}
\end{figure}

\begin{figure}[H]
    \centering
    \includegraphics[width=\textwidth]{Graphs/rapidez_adicion.png}
    \caption{Tiempo promedio anual en añadir el contenido a la plataforma desde su estreno }
    \label{fig:rapidez_subida}
\end{figure}

    \subsubsection{Análisis de variables categóricas (tablas de frecuencia, gráficos de barras).}

\subsection{Relaciones Iniciales}
\begin{itemize}
    \item Gráficos de dispersión (scatterplots) para identificar patrones.
    \item Observaciones preliminares sobre tendencias o correlaciones.
\end{itemize}

% --- Sección 3: Análisis de Componentes Principales (PCA) ---
\section{Análisis de Componentes Principales (PCA)}
\subsection{Proceso}
Se realizó un Análisis de Componentes Principales (PCA) para reducir la dimensionalidad del conjunto de datos, que incluye información sobre películas y series de TV. Utilizando seis variables (\textit{tipo, director, duración, fecha de estreno, fecha de adición, rating}), se han codificado las variables categóricas y estandarizado las variables numéricas. Los componentes principales obtenidos (\textit{PC1, PC2, PC3}) capturan la mayor parte de la variabilidad en los datos. Finalmente, se han graficado la varianza explicada por cada componente para visualizar la cantidad de información capturada por cada uno, observando que a partir del segundo componente principal, ya esta se vuelve insignificante.
\subsection{Gráficos}
\begin{figure}[H]
    \centering
    \includegraphics[width=\textwidth]{Graphs/PC_peliculas.png}
    \caption{Varianza explicada por componentes principales (Películas)}
    \label{fig:PC_peliculas}
\end{figure}

\begin{figure}[H]
    \centering
    \includegraphics[width=\textwidth]{Graphs/PC_series.png}
    \caption{Varianza explicada por componentes principales (Series)}
    \label{fig:PC_series}
\end{figure}

\begin{figure}[H]
    \centering
    \includegraphics[width=\textwidth]{Graphs/scatterplot.png}
    \caption{Gráfico de puntos para los componentes principales}
    \label{fig:PC_series}
\end{figure}

Dado que el tercer componente principal no captura una porción relativamente influyente de varianza explicada, se ha propuesto hacer el gráfico considerando solo los dos primeros PC. En el caso de las series se da la peculiaridad de que aparece un número anormalmente escaso en dicho gráfico, esto es debido a que no se consideran aquellas que poseen un valor nulo en alguno de los campos correspondientes a las variables que incluimos en el PCA. Esta es una deficiencia del conjunto de datos estudiado.

\section{Test de Normalidad}
En este análisis, se realizó un test de Kolmogorov-Smirnov para evaluar si la muestra de datos sigue una distribución normal. Tras llevar a cabo el test, se concluyó que los datos no siguen una distribución normal, lo cual nos indica que los datos presentan una desviación significativa de la distribución normal teórica.

\begin{figure}[H]
    \centering
    \includegraphics[width=\textwidth]{Graphs/dist_duracion_peliculas.png}
    \caption{Test de normalidad (Peliculas)}
    \label{fig:dist_duracion_peliculas}
\end{figure}

\begin{figure}[H]
    \centering
    \includegraphics[width=\textwidth]{Graphs/dist_duracion_series.png}
    \caption{Test de normalidad (Series)}
    \label{fig:dist_duracion_series}
\end{figure}



% --- Sección 4: Formulación de Hipótesis ---
\section{Formulación de Hipótesis}

\subsection{ Duración en películas}
 Objetivo: Comprobar si al menos el 70\% de las películas tienen una duración mayor a 80 minutos.

\begin{itemize}
	\item Hipótesis nula : La proporción es menor o igual a 0.70 .
	\item Hipótesis alternativa:La proporción es mayor a 0.70.
\end{itemize}

Dado que la media muestral de las peliculas es mayor a 80 minutos , es lógico pensar que gran parte de las peliculas de Netlfix tienen una duración superior a 80 minutos.
\newline
\newline
Luego de realizada la prueba de hipotesis(ver en el archivo Hipotesis.ipynb) se rechaza la hipotesis nula, por lo que podemos asegurar en un 95\%  que más del 70\% de las películas de Netflix tienen una duración superior a 80 minutos.


\subsection{Rating en películas y series}
\begin{itemize}
	\item Hipótesis nula :  El rating más común en el contenido  de EE.UU. no es TV-MA.
	\item Hipótesis alternativa: El rating más común de EE.UU. es TV-MA.
\end{itemize}
En el análisis exploratorio de datos se pudo observar que en la muestra  la categoría mas común tanto en peliculas como en series es TV-MA.Siendo Estados Unidos es el pais con mayor contenido en peliculas y series,se desea comprobar si el rating mas comun en su contenido es TV-MA.
\newline
\newline
Luego de realizada la prueba de hipotesis(ver en el archivo Hipotesis.ipynb) se rechaza la hipotesis nula, por lo que podemos asegurar en un 95\%  que  el rating mas comun en EEUU es TV-MA


\subsection{Prueba de los castings en EEUU e India}
Si observamos los gráficos , notamos que los actores mas comunes de las películas son de nacionalidad India, aún cuando el pais con mas peliculas es Estados Unidos (seguido de India).Esto sugiere diferencias en las industrias cinematográficas de ambos países, posiblemente en términos de diversidad y frecuencia de aparición de los actores en las producciones.

\begin{itemize}
	\item Hipótesis nula : No hay diferencia significativa en la frecuencia promedio de aparición de actores entre las películas de India y las de Estados Unidos.
	\item Hipótesis alternativa:Los actores en las películas de India tienen una frecuencia promedio de aparición mayor que los actores en las películas de Estados Unidos.
\end{itemize}
Luego de realizada la prueba de hipotesis(ver en el archivo Hipotesis.ipynb) se rechaza la hipotesis nula, por lo que podemos asegurar en un 95\%  que los actores en películas de India aparecen en más películas en promedio que los actores en películas de EE.UU.
\newline
\newline
nota: Algo similar sucede con las series ,aun cuando Estados Unidos es el pais con mas series, los actores mas comunes son de nacionalidad  japonesa.





\subsection{ Categorias Más Común en Películas}
Queremos determinar si "International Movies" es la categoría más frecuente en las películas del catálogo de Netflix, y si su frecuencia es significativamente mayor que la esperada si todas las categorías fueran igualmente comunes.
\begin{itemize}
	
	\item Hipótesis nula : La categoría "International Movies" no es la categoría más común entre las películas; su frecuencia no difiere significativamente de las demás categorías.
	\item Hipótesis alternativa:La categoría "International Movies" es la categoría más común entre las películas; su frecuencia es significativamente mayor que la de otras categorías.
\end{itemize}
Luego de realizada la prueba de hipotesis(ver en el archivo Hipotesis.ipynb) se rechaza la hipotesis nula, por lo que podemos asegurar en un 95\%  que  La categoría "International Movies" es la categoría más común entre las películas.


% --- Sección 5: Análisis de Correlación ---
\section{Análisis de Correlación}
\subsection{Matriz de Correlación}
Para la matriz de correlación , se utilizaron las siguientes variables (luego de las transformaciones necesarias):
\begin{itemize}
	\item duration :Convertiremos la columna duration a variables numéricas separadas para películas y series.
	\item date\_added :Convertiremos las fechas a un formato numérico (días desde una fecha de referencia).
	\item rating: Asignaremos valores numéricos a las clasificaciones por edad.
	\item is\_movie: Indicador binario (1 si es película, 0 si es serie).
\end{itemize}

\subsubsection{Películas}
\begin{figure}[H]
	\centering
	\includegraphics[width=\textwidth]{Graphs/matriz_correlacion_peliculas.png}
	\caption{Tiempo promedio anual en añadir el contenido a la plataforma desde su estreno }
	\label{fig:matriz_correlacion_peliculas}
\end{figure}

\subsubsection{Series}
\begin{figure}[H]
	\centering
	\includegraphics[width=\textwidth]{Graphs/matriz_correlacion_series.png}
	\caption{Tiempo promedio anual en añadir el contenido a la plataforma desde su estreno }
	\label{fig:matriz_correlacion_series}
\end{figure}



\subsection{Interpretación de Correlaciones}
\begin{itemize}
    \item Valores cercanos a 1: Fuerte correlación positiva. Cuando una variable aumenta, la otra también tiende a aumentar.
    \item Valores cercanos a -1: Fuerte correlación negativa. Cuando una variable aumenta, la otra tiende a disminuir.
    \item Valores cercanos a 0: No hay correlación lineal. Las variables no tienen una relación lineal clara.
\end{itemize}

% --- Sección 6: Regresión Lineal ---
\section{Regresión Lineal}
\subsection{Selección de Variables}
\begin{itemize}
    \item Variables independientes y dependientes seleccionadas.
    \item Justificación de la selección basada en el EDA, PCA y correlación.
\end{itemize}

\subsection{División del Dataset}
\begin{itemize}
    \item Proporción de datos de entrenamiento y prueba (ej. 80\%-20\%).
\end{itemize}

\subsection{Ajuste del Modelo}
\begin{itemize}
    \item Descripción del modelo de regresión lineal ajustado.
    \item Ecuación del modelo.
\end{itemize}

\subsection{Evaluación del Modelo}
\begin{itemize}
    \item Métricas de rendimiento (R², MSE, MAE).
    \item Análisis de residuos (normalidad, homocedasticidad, independencia).
\end{itemize}

\subsection{Interpretación de Resultados}
\begin{itemize}
    \item Impacto de cada variable independiente en la dependiente.
    \item Conclusiones basadas en los coeficientes del modelo.
\end{itemize}

% --- Sección 7: Validación y Conclusiones ---
\section{Validación y Conclusiones}
\subsection{Validación Cruzada}
\begin{itemize}
    \item Descripción del proceso de validación cruzada.
    \item Resultados de la validación.
\end{itemize}

\subsection{Conclusiones}
\begin{itemize}
    \item Resumen de los hallazgos principales.
    \item Limitaciones del análisis.
    \item Recomendaciones basadas en los resultados.
\end{itemize}

% --- Sección 8: Apéndices ---
\section{Apéndices}
\subsection{Código Utilizado}
Incluye el código utilizado para el análisis (si es relevante).

\subsection{Tablas y Figuras Adicionales}
\begin{itemize}
    \item Tablas y gráficos que no se incluyeron en el cuerpo principal pero que son relevantes.
\end{itemize}

% --- Referencias ---
\section*{Referencias}
\begin{itemize}
    \item Libros, artículos o recursos utilizados para el análisis.
\end{itemize}

\end{document}